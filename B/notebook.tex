
% Default to the notebook output style

    


% Inherit from the specified cell style.




    
\documentclass[11pt]{article}

    
    
    \usepackage[T1]{fontenc}
    % Nicer default font (+ math font) than Computer Modern for most use cases
    \usepackage{mathpazo}

    % Basic figure setup, for now with no caption control since it's done
    % automatically by Pandoc (which extracts ![](path) syntax from Markdown).
    \usepackage{graphicx}
    % We will generate all images so they have a width \maxwidth. This means
    % that they will get their normal width if they fit onto the page, but
    % are scaled down if they would overflow the margins.
    \makeatletter
    \def\maxwidth{\ifdim\Gin@nat@width>\linewidth\linewidth
    \else\Gin@nat@width\fi}
    \makeatother
    \let\Oldincludegraphics\includegraphics
    % Set max figure width to be 80% of text width, for now hardcoded.
    \renewcommand{\includegraphics}[1]{\Oldincludegraphics[width=.8\maxwidth]{#1}}
    % Ensure that by default, figures have no caption (until we provide a
    % proper Figure object with a Caption API and a way to capture that
    % in the conversion process - todo).
    \usepackage{caption}
    \DeclareCaptionLabelFormat{nolabel}{}
    \captionsetup{labelformat=nolabel}

    \usepackage{adjustbox} % Used to constrain images to a maximum size 
    \usepackage{xcolor} % Allow colors to be defined
    \usepackage{enumerate} % Needed for markdown enumerations to work
    \usepackage{geometry} % Used to adjust the document margins
    \usepackage{amsmath} % Equations
    \usepackage{amssymb} % Equations
    \usepackage{textcomp} % defines textquotesingle
    % Hack from http://tex.stackexchange.com/a/47451/13684:
    \AtBeginDocument{%
        \def\PYZsq{\textquotesingle}% Upright quotes in Pygmentized code
    }
    \usepackage{upquote} % Upright quotes for verbatim code
    \usepackage{eurosym} % defines \euro
    \usepackage[mathletters]{ucs} % Extended unicode (utf-8) support
    \usepackage[utf8x]{inputenc} % Allow utf-8 characters in the tex document
    \usepackage{fancyvrb} % verbatim replacement that allows latex
    \usepackage{grffile} % extends the file name processing of package graphics 
                         % to support a larger range 
    % The hyperref package gives us a pdf with properly built
    % internal navigation ('pdf bookmarks' for the table of contents,
    % internal cross-reference links, web links for URLs, etc.)
    \usepackage{hyperref}
    \usepackage{longtable} % longtable support required by pandoc >1.10
    \usepackage{booktabs}  % table support for pandoc > 1.12.2
    \usepackage[inline]{enumitem} % IRkernel/repr support (it uses the enumerate* environment)
    \usepackage[normalem]{ulem} % ulem is needed to support strikethroughs (\sout)
                                % normalem makes italics be italics, not underlines
    

    
    
    % Colors for the hyperref package
    \definecolor{urlcolor}{rgb}{0,.145,.698}
    \definecolor{linkcolor}{rgb}{.71,0.21,0.01}
    \definecolor{citecolor}{rgb}{.12,.54,.11}

    % ANSI colors
    \definecolor{ansi-black}{HTML}{3E424D}
    \definecolor{ansi-black-intense}{HTML}{282C36}
    \definecolor{ansi-red}{HTML}{E75C58}
    \definecolor{ansi-red-intense}{HTML}{B22B31}
    \definecolor{ansi-green}{HTML}{00A250}
    \definecolor{ansi-green-intense}{HTML}{007427}
    \definecolor{ansi-yellow}{HTML}{DDB62B}
    \definecolor{ansi-yellow-intense}{HTML}{B27D12}
    \definecolor{ansi-blue}{HTML}{208FFB}
    \definecolor{ansi-blue-intense}{HTML}{0065CA}
    \definecolor{ansi-magenta}{HTML}{D160C4}
    \definecolor{ansi-magenta-intense}{HTML}{A03196}
    \definecolor{ansi-cyan}{HTML}{60C6C8}
    \definecolor{ansi-cyan-intense}{HTML}{258F8F}
    \definecolor{ansi-white}{HTML}{C5C1B4}
    \definecolor{ansi-white-intense}{HTML}{A1A6B2}

    % commands and environments needed by pandoc snippets
    % extracted from the output of `pandoc -s`
    \providecommand{\tightlist}{%
      \setlength{\itemsep}{0pt}\setlength{\parskip}{0pt}}
    \DefineVerbatimEnvironment{Highlighting}{Verbatim}{commandchars=\\\{\}}
    % Add ',fontsize=\small' for more characters per line
    \newenvironment{Shaded}{}{}
    \newcommand{\KeywordTok}[1]{\textcolor[rgb]{0.00,0.44,0.13}{\textbf{{#1}}}}
    \newcommand{\DataTypeTok}[1]{\textcolor[rgb]{0.56,0.13,0.00}{{#1}}}
    \newcommand{\DecValTok}[1]{\textcolor[rgb]{0.25,0.63,0.44}{{#1}}}
    \newcommand{\BaseNTok}[1]{\textcolor[rgb]{0.25,0.63,0.44}{{#1}}}
    \newcommand{\FloatTok}[1]{\textcolor[rgb]{0.25,0.63,0.44}{{#1}}}
    \newcommand{\CharTok}[1]{\textcolor[rgb]{0.25,0.44,0.63}{{#1}}}
    \newcommand{\StringTok}[1]{\textcolor[rgb]{0.25,0.44,0.63}{{#1}}}
    \newcommand{\CommentTok}[1]{\textcolor[rgb]{0.38,0.63,0.69}{\textit{{#1}}}}
    \newcommand{\OtherTok}[1]{\textcolor[rgb]{0.00,0.44,0.13}{{#1}}}
    \newcommand{\AlertTok}[1]{\textcolor[rgb]{1.00,0.00,0.00}{\textbf{{#1}}}}
    \newcommand{\FunctionTok}[1]{\textcolor[rgb]{0.02,0.16,0.49}{{#1}}}
    \newcommand{\RegionMarkerTok}[1]{{#1}}
    \newcommand{\ErrorTok}[1]{\textcolor[rgb]{1.00,0.00,0.00}{\textbf{{#1}}}}
    \newcommand{\NormalTok}[1]{{#1}}
    
    % Additional commands for more recent versions of Pandoc
    \newcommand{\ConstantTok}[1]{\textcolor[rgb]{0.53,0.00,0.00}{{#1}}}
    \newcommand{\SpecialCharTok}[1]{\textcolor[rgb]{0.25,0.44,0.63}{{#1}}}
    \newcommand{\VerbatimStringTok}[1]{\textcolor[rgb]{0.25,0.44,0.63}{{#1}}}
    \newcommand{\SpecialStringTok}[1]{\textcolor[rgb]{0.73,0.40,0.53}{{#1}}}
    \newcommand{\ImportTok}[1]{{#1}}
    \newcommand{\DocumentationTok}[1]{\textcolor[rgb]{0.73,0.13,0.13}{\textit{{#1}}}}
    \newcommand{\AnnotationTok}[1]{\textcolor[rgb]{0.38,0.63,0.69}{\textbf{\textit{{#1}}}}}
    \newcommand{\CommentVarTok}[1]{\textcolor[rgb]{0.38,0.63,0.69}{\textbf{\textit{{#1}}}}}
    \newcommand{\VariableTok}[1]{\textcolor[rgb]{0.10,0.09,0.49}{{#1}}}
    \newcommand{\ControlFlowTok}[1]{\textcolor[rgb]{0.00,0.44,0.13}{\textbf{{#1}}}}
    \newcommand{\OperatorTok}[1]{\textcolor[rgb]{0.40,0.40,0.40}{{#1}}}
    \newcommand{\BuiltInTok}[1]{{#1}}
    \newcommand{\ExtensionTok}[1]{{#1}}
    \newcommand{\PreprocessorTok}[1]{\textcolor[rgb]{0.74,0.48,0.00}{{#1}}}
    \newcommand{\AttributeTok}[1]{\textcolor[rgb]{0.49,0.56,0.16}{{#1}}}
    \newcommand{\InformationTok}[1]{\textcolor[rgb]{0.38,0.63,0.69}{\textbf{\textit{{#1}}}}}
    \newcommand{\WarningTok}[1]{\textcolor[rgb]{0.38,0.63,0.69}{\textbf{\textit{{#1}}}}}
    
    
    % Define a nice break command that doesn't care if a line doesn't already
    % exist.
    \def\br{\hspace*{\fill} \\* }
    % Math Jax compatability definitions
    \def\gt{>}
    \def\lt{<}
    % Document parameters
    \title{Lecture4}
    
    
    

    % Pygments definitions
    
\makeatletter
\def\PY@reset{\let\PY@it=\relax \let\PY@bf=\relax%
    \let\PY@ul=\relax \let\PY@tc=\relax%
    \let\PY@bc=\relax \let\PY@ff=\relax}
\def\PY@tok#1{\csname PY@tok@#1\endcsname}
\def\PY@toks#1+{\ifx\relax#1\empty\else%
    \PY@tok{#1}\expandafter\PY@toks\fi}
\def\PY@do#1{\PY@bc{\PY@tc{\PY@ul{%
    \PY@it{\PY@bf{\PY@ff{#1}}}}}}}
\def\PY#1#2{\PY@reset\PY@toks#1+\relax+\PY@do{#2}}

\expandafter\def\csname PY@tok@w\endcsname{\def\PY@tc##1{\textcolor[rgb]{0.73,0.73,0.73}{##1}}}
\expandafter\def\csname PY@tok@c\endcsname{\let\PY@it=\textit\def\PY@tc##1{\textcolor[rgb]{0.25,0.50,0.50}{##1}}}
\expandafter\def\csname PY@tok@cp\endcsname{\def\PY@tc##1{\textcolor[rgb]{0.74,0.48,0.00}{##1}}}
\expandafter\def\csname PY@tok@k\endcsname{\let\PY@bf=\textbf\def\PY@tc##1{\textcolor[rgb]{0.00,0.50,0.00}{##1}}}
\expandafter\def\csname PY@tok@kp\endcsname{\def\PY@tc##1{\textcolor[rgb]{0.00,0.50,0.00}{##1}}}
\expandafter\def\csname PY@tok@kt\endcsname{\def\PY@tc##1{\textcolor[rgb]{0.69,0.00,0.25}{##1}}}
\expandafter\def\csname PY@tok@o\endcsname{\def\PY@tc##1{\textcolor[rgb]{0.40,0.40,0.40}{##1}}}
\expandafter\def\csname PY@tok@ow\endcsname{\let\PY@bf=\textbf\def\PY@tc##1{\textcolor[rgb]{0.67,0.13,1.00}{##1}}}
\expandafter\def\csname PY@tok@nb\endcsname{\def\PY@tc##1{\textcolor[rgb]{0.00,0.50,0.00}{##1}}}
\expandafter\def\csname PY@tok@nf\endcsname{\def\PY@tc##1{\textcolor[rgb]{0.00,0.00,1.00}{##1}}}
\expandafter\def\csname PY@tok@nc\endcsname{\let\PY@bf=\textbf\def\PY@tc##1{\textcolor[rgb]{0.00,0.00,1.00}{##1}}}
\expandafter\def\csname PY@tok@nn\endcsname{\let\PY@bf=\textbf\def\PY@tc##1{\textcolor[rgb]{0.00,0.00,1.00}{##1}}}
\expandafter\def\csname PY@tok@ne\endcsname{\let\PY@bf=\textbf\def\PY@tc##1{\textcolor[rgb]{0.82,0.25,0.23}{##1}}}
\expandafter\def\csname PY@tok@nv\endcsname{\def\PY@tc##1{\textcolor[rgb]{0.10,0.09,0.49}{##1}}}
\expandafter\def\csname PY@tok@no\endcsname{\def\PY@tc##1{\textcolor[rgb]{0.53,0.00,0.00}{##1}}}
\expandafter\def\csname PY@tok@nl\endcsname{\def\PY@tc##1{\textcolor[rgb]{0.63,0.63,0.00}{##1}}}
\expandafter\def\csname PY@tok@ni\endcsname{\let\PY@bf=\textbf\def\PY@tc##1{\textcolor[rgb]{0.60,0.60,0.60}{##1}}}
\expandafter\def\csname PY@tok@na\endcsname{\def\PY@tc##1{\textcolor[rgb]{0.49,0.56,0.16}{##1}}}
\expandafter\def\csname PY@tok@nt\endcsname{\let\PY@bf=\textbf\def\PY@tc##1{\textcolor[rgb]{0.00,0.50,0.00}{##1}}}
\expandafter\def\csname PY@tok@nd\endcsname{\def\PY@tc##1{\textcolor[rgb]{0.67,0.13,1.00}{##1}}}
\expandafter\def\csname PY@tok@s\endcsname{\def\PY@tc##1{\textcolor[rgb]{0.73,0.13,0.13}{##1}}}
\expandafter\def\csname PY@tok@sd\endcsname{\let\PY@it=\textit\def\PY@tc##1{\textcolor[rgb]{0.73,0.13,0.13}{##1}}}
\expandafter\def\csname PY@tok@si\endcsname{\let\PY@bf=\textbf\def\PY@tc##1{\textcolor[rgb]{0.73,0.40,0.53}{##1}}}
\expandafter\def\csname PY@tok@se\endcsname{\let\PY@bf=\textbf\def\PY@tc##1{\textcolor[rgb]{0.73,0.40,0.13}{##1}}}
\expandafter\def\csname PY@tok@sr\endcsname{\def\PY@tc##1{\textcolor[rgb]{0.73,0.40,0.53}{##1}}}
\expandafter\def\csname PY@tok@ss\endcsname{\def\PY@tc##1{\textcolor[rgb]{0.10,0.09,0.49}{##1}}}
\expandafter\def\csname PY@tok@sx\endcsname{\def\PY@tc##1{\textcolor[rgb]{0.00,0.50,0.00}{##1}}}
\expandafter\def\csname PY@tok@m\endcsname{\def\PY@tc##1{\textcolor[rgb]{0.40,0.40,0.40}{##1}}}
\expandafter\def\csname PY@tok@gh\endcsname{\let\PY@bf=\textbf\def\PY@tc##1{\textcolor[rgb]{0.00,0.00,0.50}{##1}}}
\expandafter\def\csname PY@tok@gu\endcsname{\let\PY@bf=\textbf\def\PY@tc##1{\textcolor[rgb]{0.50,0.00,0.50}{##1}}}
\expandafter\def\csname PY@tok@gd\endcsname{\def\PY@tc##1{\textcolor[rgb]{0.63,0.00,0.00}{##1}}}
\expandafter\def\csname PY@tok@gi\endcsname{\def\PY@tc##1{\textcolor[rgb]{0.00,0.63,0.00}{##1}}}
\expandafter\def\csname PY@tok@gr\endcsname{\def\PY@tc##1{\textcolor[rgb]{1.00,0.00,0.00}{##1}}}
\expandafter\def\csname PY@tok@ge\endcsname{\let\PY@it=\textit}
\expandafter\def\csname PY@tok@gs\endcsname{\let\PY@bf=\textbf}
\expandafter\def\csname PY@tok@gp\endcsname{\let\PY@bf=\textbf\def\PY@tc##1{\textcolor[rgb]{0.00,0.00,0.50}{##1}}}
\expandafter\def\csname PY@tok@go\endcsname{\def\PY@tc##1{\textcolor[rgb]{0.53,0.53,0.53}{##1}}}
\expandafter\def\csname PY@tok@gt\endcsname{\def\PY@tc##1{\textcolor[rgb]{0.00,0.27,0.87}{##1}}}
\expandafter\def\csname PY@tok@err\endcsname{\def\PY@bc##1{\setlength{\fboxsep}{0pt}\fcolorbox[rgb]{1.00,0.00,0.00}{1,1,1}{\strut ##1}}}
\expandafter\def\csname PY@tok@kc\endcsname{\let\PY@bf=\textbf\def\PY@tc##1{\textcolor[rgb]{0.00,0.50,0.00}{##1}}}
\expandafter\def\csname PY@tok@kd\endcsname{\let\PY@bf=\textbf\def\PY@tc##1{\textcolor[rgb]{0.00,0.50,0.00}{##1}}}
\expandafter\def\csname PY@tok@kn\endcsname{\let\PY@bf=\textbf\def\PY@tc##1{\textcolor[rgb]{0.00,0.50,0.00}{##1}}}
\expandafter\def\csname PY@tok@kr\endcsname{\let\PY@bf=\textbf\def\PY@tc##1{\textcolor[rgb]{0.00,0.50,0.00}{##1}}}
\expandafter\def\csname PY@tok@bp\endcsname{\def\PY@tc##1{\textcolor[rgb]{0.00,0.50,0.00}{##1}}}
\expandafter\def\csname PY@tok@fm\endcsname{\def\PY@tc##1{\textcolor[rgb]{0.00,0.00,1.00}{##1}}}
\expandafter\def\csname PY@tok@vc\endcsname{\def\PY@tc##1{\textcolor[rgb]{0.10,0.09,0.49}{##1}}}
\expandafter\def\csname PY@tok@vg\endcsname{\def\PY@tc##1{\textcolor[rgb]{0.10,0.09,0.49}{##1}}}
\expandafter\def\csname PY@tok@vi\endcsname{\def\PY@tc##1{\textcolor[rgb]{0.10,0.09,0.49}{##1}}}
\expandafter\def\csname PY@tok@vm\endcsname{\def\PY@tc##1{\textcolor[rgb]{0.10,0.09,0.49}{##1}}}
\expandafter\def\csname PY@tok@sa\endcsname{\def\PY@tc##1{\textcolor[rgb]{0.73,0.13,0.13}{##1}}}
\expandafter\def\csname PY@tok@sb\endcsname{\def\PY@tc##1{\textcolor[rgb]{0.73,0.13,0.13}{##1}}}
\expandafter\def\csname PY@tok@sc\endcsname{\def\PY@tc##1{\textcolor[rgb]{0.73,0.13,0.13}{##1}}}
\expandafter\def\csname PY@tok@dl\endcsname{\def\PY@tc##1{\textcolor[rgb]{0.73,0.13,0.13}{##1}}}
\expandafter\def\csname PY@tok@s2\endcsname{\def\PY@tc##1{\textcolor[rgb]{0.73,0.13,0.13}{##1}}}
\expandafter\def\csname PY@tok@sh\endcsname{\def\PY@tc##1{\textcolor[rgb]{0.73,0.13,0.13}{##1}}}
\expandafter\def\csname PY@tok@s1\endcsname{\def\PY@tc##1{\textcolor[rgb]{0.73,0.13,0.13}{##1}}}
\expandafter\def\csname PY@tok@mb\endcsname{\def\PY@tc##1{\textcolor[rgb]{0.40,0.40,0.40}{##1}}}
\expandafter\def\csname PY@tok@mf\endcsname{\def\PY@tc##1{\textcolor[rgb]{0.40,0.40,0.40}{##1}}}
\expandafter\def\csname PY@tok@mh\endcsname{\def\PY@tc##1{\textcolor[rgb]{0.40,0.40,0.40}{##1}}}
\expandafter\def\csname PY@tok@mi\endcsname{\def\PY@tc##1{\textcolor[rgb]{0.40,0.40,0.40}{##1}}}
\expandafter\def\csname PY@tok@il\endcsname{\def\PY@tc##1{\textcolor[rgb]{0.40,0.40,0.40}{##1}}}
\expandafter\def\csname PY@tok@mo\endcsname{\def\PY@tc##1{\textcolor[rgb]{0.40,0.40,0.40}{##1}}}
\expandafter\def\csname PY@tok@ch\endcsname{\let\PY@it=\textit\def\PY@tc##1{\textcolor[rgb]{0.25,0.50,0.50}{##1}}}
\expandafter\def\csname PY@tok@cm\endcsname{\let\PY@it=\textit\def\PY@tc##1{\textcolor[rgb]{0.25,0.50,0.50}{##1}}}
\expandafter\def\csname PY@tok@cpf\endcsname{\let\PY@it=\textit\def\PY@tc##1{\textcolor[rgb]{0.25,0.50,0.50}{##1}}}
\expandafter\def\csname PY@tok@c1\endcsname{\let\PY@it=\textit\def\PY@tc##1{\textcolor[rgb]{0.25,0.50,0.50}{##1}}}
\expandafter\def\csname PY@tok@cs\endcsname{\let\PY@it=\textit\def\PY@tc##1{\textcolor[rgb]{0.25,0.50,0.50}{##1}}}

\def\PYZbs{\char`\\}
\def\PYZus{\char`\_}
\def\PYZob{\char`\{}
\def\PYZcb{\char`\}}
\def\PYZca{\char`\^}
\def\PYZam{\char`\&}
\def\PYZlt{\char`\<}
\def\PYZgt{\char`\>}
\def\PYZsh{\char`\#}
\def\PYZpc{\char`\%}
\def\PYZdl{\char`\$}
\def\PYZhy{\char`\-}
\def\PYZsq{\char`\'}
\def\PYZdq{\char`\"}
\def\PYZti{\char`\~}
% for compatibility with earlier versions
\def\PYZat{@}
\def\PYZlb{[}
\def\PYZrb{]}
\makeatother


    % Exact colors from NB
    \definecolor{incolor}{rgb}{0.0, 0.0, 0.5}
    \definecolor{outcolor}{rgb}{0.545, 0.0, 0.0}



    
    % Prevent overflowing lines due to hard-to-break entities
    \sloppy 
    % Setup hyperref package
    \hypersetup{
      breaklinks=true,  % so long urls are correctly broken across lines
      colorlinks=true,
      urlcolor=urlcolor,
      linkcolor=linkcolor,
      citecolor=citecolor,
      }
    % Slightly bigger margins than the latex defaults
    
    \geometry{verbose,tmargin=1in,bmargin=1in,lmargin=1in,rmargin=1in}
    
    

    \begin{document}
    
    
    \maketitle
    
    

    
    August 30, 2018

\textbf{Just basic review from last lecture and notations}

First, let DA = Deferred Acceptance Algorithm.

Brief Review of DA:

Set of acceptable partners for \(i\):
\(A(i) = \{j \in J: j \succeq_i i \}\)

\(X \subset J\)
\(\max_i X = \{j \in X: j \succeq_i j' \forall j' \in X \}\)

if \(X \ne \emptyset\) then \(\max_i X = \emptyset\)

DA (T-proposing)

\textbf{Round 1} \(A^1(t) = A(t)\) set for each \(t \in T\) set of
acceptable agents for agent t

\(\hat{P}^1: T \rightarrow B \cup \{\emptyset\} : \hat{P}^1(t) \in \max_t A^1(t)\)
specify proposal to some maximal agent that is acceptable to her (t)

Set for each \(b \in B\)
\(P^1(b) = (\hat{P}^1)^{-1}(\{b\}) \cap A(b) = \{t \in A(b): \hat{P}^1(t) = b\}\)
Induce set of proposals for b agents: the set of t's that are acceptable
to b. Could be empty

\(\hat{\mu}^1: B \rightarrow T \cup \{\emptyset\}\) and
\(\hat{mu}^1 \in \max_b P^1(b)\) Given this, the b agents will choose a
maximal agents from the set of his round one proposals.

    \(A^{k+1}(t) =\) either \textbf{(1)} or \textbf{(2)}

\textbf{(1)} \(A^k(t)\) if \(\hat{\mu}^k(\hat{P}^k) = t\) meaning t
proposed at k to \(b = \hat{P}^k(t)\) and b accepted t at \(k\).

But, he if rejected the offer, I throw him out of the running:

\textbf{(2)} \(A^k(t) \setminus \{\hat{P}^k(t)\}\) Takeaway particular b
from set if denied in step 1

if

\(\hat{\mu}^k(\hat{P}^k(t)) \ne t\)

If I was accepted on round \(k\), then I offer the same offer to the
same person on round \(k+1\).

\(\hat{P}^{k+1}(t) = \hat{P}^k\) if it is the case that
\(\hat{P}^k(t) \in A^{k+1}(t)\)

\(\hat{P}^{k+1}(t) \in \max_t A^{k+1}(t)\) if it is the case
\(\hat{P}^k(t) \notin A^{k+1}(t)\)

The set of proposals that b has on round \(k+1\):

\(P^{k+1}(b) = (\hat{P}^{k+1})^{-1} (\{b\}) \cap A(b)\)

Which is equivalent to:

\[\{t \in T: \hat{P}^{k+1}(t) = b: t \in A(b)\]

    I hold on to exactly the same offer if it is the case that b is still a
maximizer for me.

\[\hat{\mu}^{k+1}: B \rightarrow T \cup \{B\}\]

\(\hat{\mu}^{k+1}(b) = \hat{\mu}^{k}(b)\) if it is the case that
\(\hat{\mu}^{k+1}(b) \in \max_b P^{k+1}(b)\)

And

\(\hat{\mu}^{k+1}(b) = \max_b P^{k+1}(b)\) if it is the case that
\(\hat{\mu}^{k+1}(b) \notin \max_b P^{k+1}(b)\)

    \subsection{Matching}\label{matching}

We defined a matching from agents of one side of a market to another
side of the market such that: \(\mu :(T \cup B) \rightarrow (T \cup B)\)

If you go through the rounds the t agents make new proposals. As they
make new proposals, it is worse for t agents. As t makes new proposals,
it is also better for the b agents since they get more options which
might be preferred.

\subsubsection{Lemma 2}\label{lemma-2}

\begin{enumerate}
\def\labelenumi{\arabic{enumi}.}
\tightlist
\item
  For eack \(k\) and each \(t \in T\),
  \(\hat{P}^k(t) \succeq_t \hat{P}^{k+1}(t).\)
\item
  For each \(k\) and each \(b \in B\),
  \(\hat{mu}^{k+1} \succeq_b \hat{\mu}^k(b).\)
\end{enumerate}

\textbf{Proof}

\begin{enumerate}
\def\labelenumi{(\arabic{enumi})}
\item
\end{enumerate}

Recall: \(A^{k+1}(t) \subset A^(t)\)

\(\hat{P}^{k+1}(t) \in \max_t A^{k+1}(t)\)

\(\hat{P}^{k}(t) \in \max_t A^{k}(t)\)

\(\implies \hat{P}^{k}(t) \succeq_t \hat{P}^{k+1}(t)\)

    \begin{enumerate}
\def\labelenumi{(\arabic{enumi})}
\setcounter{enumi}{1}
\item
\end{enumerate}

Suppose: \(\hat{\mu}^k(b) = t\) and so \(\hat{P}^{k}(t)=b\)

\(A^{k+1} = A^k(t) \implies \hat{P}^{k+1}(t) = \hat{P}^{k}(t) = b\)

\(\implies t \in \hat{P}^{k+1}(t)\)

if \(\hat{mu}^{k+1}(b) = t\) then certainly
\(\hat{mu}^{k+1}(b) \succeq_b \hat{mu}^{k}(b)\)

if \(\hat{mu}^{k+1}(b) \ne t\) then
\(\hat{mu}^{k+1}(b) \succeq_b t = \hat{\mu}^k(b)\)

    \subsubsection{Lemma 3}\label{lemma-3}

Fix \(b \ne b'\) If \(\hat{mu}^{k}(b) = \hat{mu}^{k}(b')\) then
\(\hat{mu}^{k}(b) = \hat{mu}^{k}(b') = \emptyset\)

\textbf{Proof:}\_

Fix \(b \ne b'\)

\(\hat{mu}^{k}(b) \in P^k(b)\)

\(\hat{mu}^{k}(b') \in P^k(b')\)

\(P^k(b) \cap P^k(b') = \emptyset\)

If \(t \in P^k(b) \cap P^k(b')\) then \(\hat{P}^k(t) = b = b'\)

This is a contradiction, since a function is mapping more than one thing
to another (like one x to two y's). \(P^k\) is a function.

    By Lemma 1 (last class) there exits a \(k < \infty\) such that
\(\hat{mu}^{K} = \hat{mu}^{k}\) for all \(k \ge K\).

\(\mu^*: (T \cup B) \rightarrow (T \cup B)\)

\begin{itemize}
\tightlist
\item
  \(\mu^*(b) < \hat{\mu}^k(b)\) for each \(b \in B\) such that
  \(\hat{\mu}^k(b) \in T\)
\item
  \(\mu^*(b) = b\) for each \(b \in B\) such that
  \(\hat{\mu}^k(b) = \emptyset\)
\end{itemize}

By Lemma 3 for each \(t \in T\), there is at most 1 \(b\) such that
\(\hat{\mu}^k(b) = t\).

Case 1: \(\hat{\mu}^*(t) = b\) if \(b = (\hat{\mu}^k)^-1(t)\)

Case 2: \(\mu^*(t) = t\) otherwise

    So, we have constructed a matching which leads to Lemma 4:

\subsubsection{Lemma 4}\label{lemma-4}

The matching \(\mu^*: (T \cup B) \rightarrow (T \cup B)\) is stable.

\textbf{Proof:}

Need to show two things: 1. Matching is indivudally rations:
\(\mu^*(i) \succeq_i i\) 2. There is no blocking pair \((t,b)\).

Step 1 for the t agent:

\textbf{(1.t)}

It suffices to show that for each \(t \in T\) that is matched
\(\mu^*(b) \in B\) there is \(\mu^*(t) \in A(t)\)

\(\mu^*(t) = b \implies \hat{P}^k(t) = b \implies b \in A^k(t) \subset A(t)\)

Step 1 for the b agent:

\textbf{(1.b)}

It suffices to show that for each \(b \in B\) with \(\mu^*(b) \in T\),
\(\mu^*(b) \in A(b)\)

Notice by construction that
\(\mu^*(b) = \hat{\mu}^k(b) \in P^k(b) \subset A(b)\)

    \textbf{(2)}

Fix (t, b) and suppose contra hypothesis that (t,b) blocks \(\mu^*\)

This means \(b \succeq_t \mu^*(t)\) and \(t \succeq_b \mu^*(b)\).

There exists some \(k < K\) such that \(\hat{P}^k(t) = b\) and
\(\hat{\mu}^k(b) \ne t\).

then:

\(\mu^*(b) = \mu^*(b) \succeq_b \mu^k(b) \succeq_b t\) This follows from
Lemma 2.

This implies \(\mu^*(b) \succeq_b t\) which contradicts that \(t\) forms
a blocking pair.

    Algorithm tells us how to construct a stable match. Presuming that
designer knows all inputs and preferences. Program a computer to make
all those matches. However, we will talk later about when designer does
not know preferences. But, now, we assume so, and have shown this is
possible to construct.

No one has incentive to deviate. Or we hope. We want them to be
normatively desirable. Are they good for society? There are lots of
different criteria of how we understand normatively desirable matches.

    \subsection{Are Stable Matchings Normatively
desirable?}\label{are-stable-matchings-normatively-desirable}

\begin{itemize}
\item
  Criteria 1: \emph{Are stable matches Pareto efficient?}
\item
  Stable matches tend to be pareto efficient (hw problem to show)
\item
  Pareto efficient is a state of allocation of resources from which it
  is impossible to reallocate so as to make any one individual or
  preference criterion better off without making at least one individual
  or preference criterion worse off.
\item
  Criteria 2: \emph{selecting the best stable matching among all the
  stable matchings possible}
\end{itemize}

\textbf{Example}: (from last class)

    Imagine an environment with, say, two firms \(T\) and two employees
\(B\):

\(T = \{t_1, t_2\}\)

\(B = \{b_1, b_2\}\)

Preferences of \(T\):

\(t_1: \space b_1 \succsim_{t_1} b_2 \succsim_{t_1} t_1\)

\(t_2: \space b_2 \succsim_{t_2} b_1 \succsim_{t_2} t_2\)

Preferences of \(B\):

\(b_1: \space t_2 \succsim_{b_1} t_1 \succsim_{b_1} b_1\)

\(b_2: \space t_1 \succsim_{b_2} t_2 \succsim_{b_2} b_2\)

    Recall that there were two stable matches:

\begin{enumerate}
\def\labelenumi{(\arabic{enumi})}
\tightlist
\item
  All B agents get best match (B-proposing DA algorithm)
\end{enumerate}

\(\mu(t_1) = b_2\) and \(\mu(t_2) = b_1\)

\(\mu(b_2) = t_1\) and \(\mu(b_1) = t_2\)

\begin{enumerate}
\def\labelenumi{(\arabic{enumi})}
\setcounter{enumi}{1}
\tightlist
\item
  All T agents get best match. (T-proposing DA algorithm)
\end{enumerate}

\(\mu(t_1) = b_1\) and \(\mu(t_2) = b_2\)

\(\mu(b_1) = t_1\) and \(\mu(b_2) = t_2\)

    \subsection{Side of the Market
Preferences}\label{side-of-the-market-preferences}

If I think about the \(I\) proposing DA algorithm:

\(\mu_{ID}: (T \cup B) \rightarrow (T \cup B)\)

Introduce an order: \(\ge_I\) such that \(\mu \ge_I \mu'\) if, for each
\(i \in I: \mu(i) \succeq_i \mu'(i)\)

Write:

\(\mu >_I \mu'\) if \(\mu \ge_I \mu'\) and there exists some \(i \in I\)
such that \(\mu(i) \succ_i \mu'(i)\)

I will call \(\mu:(T \cup B) \rightarrow (T \cup B)\) an
\textbf{I-Optimal} stable match if:

\begin{enumerate}
\def\labelenumi{\arabic{enumi}.}
\tightlist
\item
  \(\mu\) is stable.
\item
  \(\mu \ge_I \mu'\) for all other stable matches \(\mu\).
\end{enumerate}

    \subsubsection{Proposition (Gale - Shapley
1962)}\label{proposition-gale---shapley-1962}

Suppose preferences are strict. Then for each \(I \in \{T, B\}\):

\(\mu_{ID}: (T \cup B) \rightarrow (T \cup B)\) is an I-optimal stable
match.

The proof and example will be in the next class.


    % Add a bibliography block to the postdoc
    
    
    
    \end{document}
