
% Default to the notebook output style

    


% Inherit from the specified cell style.




    
\documentclass[11pt]{article}

    
    
    \usepackage[T1]{fontenc}
    % Nicer default font (+ math font) than Computer Modern for most use cases
    \usepackage{mathpazo}

    % Basic figure setup, for now with no caption control since it's done
    % automatically by Pandoc (which extracts ![](path) syntax from Markdown).
    \usepackage{graphicx}
    % We will generate all images so they have a width \maxwidth. This means
    % that they will get their normal width if they fit onto the page, but
    % are scaled down if they would overflow the margins.
    \makeatletter
    \def\maxwidth{\ifdim\Gin@nat@width>\linewidth\linewidth
    \else\Gin@nat@width\fi}
    \makeatother
    \let\Oldincludegraphics\includegraphics
    % Set max figure width to be 80% of text width, for now hardcoded.
    \renewcommand{\includegraphics}[1]{\Oldincludegraphics[width=.8\maxwidth]{#1}}
    % Ensure that by default, figures have no caption (until we provide a
    % proper Figure object with a Caption API and a way to capture that
    % in the conversion process - todo).
    \usepackage{caption}
    \DeclareCaptionLabelFormat{nolabel}{}
    \captionsetup{labelformat=nolabel}

    \usepackage{adjustbox} % Used to constrain images to a maximum size 
    \usepackage{xcolor} % Allow colors to be defined
    \usepackage{enumerate} % Needed for markdown enumerations to work
    \usepackage{geometry} % Used to adjust the document margins
    \usepackage{amsmath} % Equations
    \usepackage{amssymb} % Equations
    \usepackage{textcomp} % defines textquotesingle
    % Hack from http://tex.stackexchange.com/a/47451/13684:
    \AtBeginDocument{%
        \def\PYZsq{\textquotesingle}% Upright quotes in Pygmentized code
    }
    \usepackage{upquote} % Upright quotes for verbatim code
    \usepackage{eurosym} % defines \euro
    \usepackage[mathletters]{ucs} % Extended unicode (utf-8) support
    \usepackage[utf8x]{inputenc} % Allow utf-8 characters in the tex document
    \usepackage{fancyvrb} % verbatim replacement that allows latex
    \usepackage{grffile} % extends the file name processing of package graphics 
                         % to support a larger range 
    % The hyperref package gives us a pdf with properly built
    % internal navigation ('pdf bookmarks' for the table of contents,
    % internal cross-reference links, web links for URLs, etc.)
    \usepackage{hyperref}
    \usepackage{longtable} % longtable support required by pandoc >1.10
    \usepackage{booktabs}  % table support for pandoc > 1.12.2
    \usepackage[inline]{enumitem} % IRkernel/repr support (it uses the enumerate* environment)
    \usepackage[normalem]{ulem} % ulem is needed to support strikethroughs (\sout)
                                % normalem makes italics be italics, not underlines
    

    
    
    % Colors for the hyperref package
    \definecolor{urlcolor}{rgb}{0,.145,.698}
    \definecolor{linkcolor}{rgb}{.71,0.21,0.01}
    \definecolor{citecolor}{rgb}{.12,.54,.11}

    % ANSI colors
    \definecolor{ansi-black}{HTML}{3E424D}
    \definecolor{ansi-black-intense}{HTML}{282C36}
    \definecolor{ansi-red}{HTML}{E75C58}
    \definecolor{ansi-red-intense}{HTML}{B22B31}
    \definecolor{ansi-green}{HTML}{00A250}
    \definecolor{ansi-green-intense}{HTML}{007427}
    \definecolor{ansi-yellow}{HTML}{DDB62B}
    \definecolor{ansi-yellow-intense}{HTML}{B27D12}
    \definecolor{ansi-blue}{HTML}{208FFB}
    \definecolor{ansi-blue-intense}{HTML}{0065CA}
    \definecolor{ansi-magenta}{HTML}{D160C4}
    \definecolor{ansi-magenta-intense}{HTML}{A03196}
    \definecolor{ansi-cyan}{HTML}{60C6C8}
    \definecolor{ansi-cyan-intense}{HTML}{258F8F}
    \definecolor{ansi-white}{HTML}{C5C1B4}
    \definecolor{ansi-white-intense}{HTML}{A1A6B2}

    % commands and environments needed by pandoc snippets
    % extracted from the output of `pandoc -s`
    \providecommand{\tightlist}{%
      \setlength{\itemsep}{0pt}\setlength{\parskip}{0pt}}
    \DefineVerbatimEnvironment{Highlighting}{Verbatim}{commandchars=\\\{\}}
    % Add ',fontsize=\small' for more characters per line
    \newenvironment{Shaded}{}{}
    \newcommand{\KeywordTok}[1]{\textcolor[rgb]{0.00,0.44,0.13}{\textbf{{#1}}}}
    \newcommand{\DataTypeTok}[1]{\textcolor[rgb]{0.56,0.13,0.00}{{#1}}}
    \newcommand{\DecValTok}[1]{\textcolor[rgb]{0.25,0.63,0.44}{{#1}}}
    \newcommand{\BaseNTok}[1]{\textcolor[rgb]{0.25,0.63,0.44}{{#1}}}
    \newcommand{\FloatTok}[1]{\textcolor[rgb]{0.25,0.63,0.44}{{#1}}}
    \newcommand{\CharTok}[1]{\textcolor[rgb]{0.25,0.44,0.63}{{#1}}}
    \newcommand{\StringTok}[1]{\textcolor[rgb]{0.25,0.44,0.63}{{#1}}}
    \newcommand{\CommentTok}[1]{\textcolor[rgb]{0.38,0.63,0.69}{\textit{{#1}}}}
    \newcommand{\OtherTok}[1]{\textcolor[rgb]{0.00,0.44,0.13}{{#1}}}
    \newcommand{\AlertTok}[1]{\textcolor[rgb]{1.00,0.00,0.00}{\textbf{{#1}}}}
    \newcommand{\FunctionTok}[1]{\textcolor[rgb]{0.02,0.16,0.49}{{#1}}}
    \newcommand{\RegionMarkerTok}[1]{{#1}}
    \newcommand{\ErrorTok}[1]{\textcolor[rgb]{1.00,0.00,0.00}{\textbf{{#1}}}}
    \newcommand{\NormalTok}[1]{{#1}}
    
    % Additional commands for more recent versions of Pandoc
    \newcommand{\ConstantTok}[1]{\textcolor[rgb]{0.53,0.00,0.00}{{#1}}}
    \newcommand{\SpecialCharTok}[1]{\textcolor[rgb]{0.25,0.44,0.63}{{#1}}}
    \newcommand{\VerbatimStringTok}[1]{\textcolor[rgb]{0.25,0.44,0.63}{{#1}}}
    \newcommand{\SpecialStringTok}[1]{\textcolor[rgb]{0.73,0.40,0.53}{{#1}}}
    \newcommand{\ImportTok}[1]{{#1}}
    \newcommand{\DocumentationTok}[1]{\textcolor[rgb]{0.73,0.13,0.13}{\textit{{#1}}}}
    \newcommand{\AnnotationTok}[1]{\textcolor[rgb]{0.38,0.63,0.69}{\textbf{\textit{{#1}}}}}
    \newcommand{\CommentVarTok}[1]{\textcolor[rgb]{0.38,0.63,0.69}{\textbf{\textit{{#1}}}}}
    \newcommand{\VariableTok}[1]{\textcolor[rgb]{0.10,0.09,0.49}{{#1}}}
    \newcommand{\ControlFlowTok}[1]{\textcolor[rgb]{0.00,0.44,0.13}{\textbf{{#1}}}}
    \newcommand{\OperatorTok}[1]{\textcolor[rgb]{0.40,0.40,0.40}{{#1}}}
    \newcommand{\BuiltInTok}[1]{{#1}}
    \newcommand{\ExtensionTok}[1]{{#1}}
    \newcommand{\PreprocessorTok}[1]{\textcolor[rgb]{0.74,0.48,0.00}{{#1}}}
    \newcommand{\AttributeTok}[1]{\textcolor[rgb]{0.49,0.56,0.16}{{#1}}}
    \newcommand{\InformationTok}[1]{\textcolor[rgb]{0.38,0.63,0.69}{\textbf{\textit{{#1}}}}}
    \newcommand{\WarningTok}[1]{\textcolor[rgb]{0.38,0.63,0.69}{\textbf{\textit{{#1}}}}}
    
    
    % Define a nice break command that doesn't care if a line doesn't already
    % exist.
    \def\br{\hspace*{\fill} \\* }
    % Math Jax compatability definitions
    \def\gt{>}
    \def\lt{<}
    % Document parameters
    \title{Lecture3}
    
    
    

    % Pygments definitions
    
\makeatletter
\def\PY@reset{\let\PY@it=\relax \let\PY@bf=\relax%
    \let\PY@ul=\relax \let\PY@tc=\relax%
    \let\PY@bc=\relax \let\PY@ff=\relax}
\def\PY@tok#1{\csname PY@tok@#1\endcsname}
\def\PY@toks#1+{\ifx\relax#1\empty\else%
    \PY@tok{#1}\expandafter\PY@toks\fi}
\def\PY@do#1{\PY@bc{\PY@tc{\PY@ul{%
    \PY@it{\PY@bf{\PY@ff{#1}}}}}}}
\def\PY#1#2{\PY@reset\PY@toks#1+\relax+\PY@do{#2}}

\expandafter\def\csname PY@tok@w\endcsname{\def\PY@tc##1{\textcolor[rgb]{0.73,0.73,0.73}{##1}}}
\expandafter\def\csname PY@tok@c\endcsname{\let\PY@it=\textit\def\PY@tc##1{\textcolor[rgb]{0.25,0.50,0.50}{##1}}}
\expandafter\def\csname PY@tok@cp\endcsname{\def\PY@tc##1{\textcolor[rgb]{0.74,0.48,0.00}{##1}}}
\expandafter\def\csname PY@tok@k\endcsname{\let\PY@bf=\textbf\def\PY@tc##1{\textcolor[rgb]{0.00,0.50,0.00}{##1}}}
\expandafter\def\csname PY@tok@kp\endcsname{\def\PY@tc##1{\textcolor[rgb]{0.00,0.50,0.00}{##1}}}
\expandafter\def\csname PY@tok@kt\endcsname{\def\PY@tc##1{\textcolor[rgb]{0.69,0.00,0.25}{##1}}}
\expandafter\def\csname PY@tok@o\endcsname{\def\PY@tc##1{\textcolor[rgb]{0.40,0.40,0.40}{##1}}}
\expandafter\def\csname PY@tok@ow\endcsname{\let\PY@bf=\textbf\def\PY@tc##1{\textcolor[rgb]{0.67,0.13,1.00}{##1}}}
\expandafter\def\csname PY@tok@nb\endcsname{\def\PY@tc##1{\textcolor[rgb]{0.00,0.50,0.00}{##1}}}
\expandafter\def\csname PY@tok@nf\endcsname{\def\PY@tc##1{\textcolor[rgb]{0.00,0.00,1.00}{##1}}}
\expandafter\def\csname PY@tok@nc\endcsname{\let\PY@bf=\textbf\def\PY@tc##1{\textcolor[rgb]{0.00,0.00,1.00}{##1}}}
\expandafter\def\csname PY@tok@nn\endcsname{\let\PY@bf=\textbf\def\PY@tc##1{\textcolor[rgb]{0.00,0.00,1.00}{##1}}}
\expandafter\def\csname PY@tok@ne\endcsname{\let\PY@bf=\textbf\def\PY@tc##1{\textcolor[rgb]{0.82,0.25,0.23}{##1}}}
\expandafter\def\csname PY@tok@nv\endcsname{\def\PY@tc##1{\textcolor[rgb]{0.10,0.09,0.49}{##1}}}
\expandafter\def\csname PY@tok@no\endcsname{\def\PY@tc##1{\textcolor[rgb]{0.53,0.00,0.00}{##1}}}
\expandafter\def\csname PY@tok@nl\endcsname{\def\PY@tc##1{\textcolor[rgb]{0.63,0.63,0.00}{##1}}}
\expandafter\def\csname PY@tok@ni\endcsname{\let\PY@bf=\textbf\def\PY@tc##1{\textcolor[rgb]{0.60,0.60,0.60}{##1}}}
\expandafter\def\csname PY@tok@na\endcsname{\def\PY@tc##1{\textcolor[rgb]{0.49,0.56,0.16}{##1}}}
\expandafter\def\csname PY@tok@nt\endcsname{\let\PY@bf=\textbf\def\PY@tc##1{\textcolor[rgb]{0.00,0.50,0.00}{##1}}}
\expandafter\def\csname PY@tok@nd\endcsname{\def\PY@tc##1{\textcolor[rgb]{0.67,0.13,1.00}{##1}}}
\expandafter\def\csname PY@tok@s\endcsname{\def\PY@tc##1{\textcolor[rgb]{0.73,0.13,0.13}{##1}}}
\expandafter\def\csname PY@tok@sd\endcsname{\let\PY@it=\textit\def\PY@tc##1{\textcolor[rgb]{0.73,0.13,0.13}{##1}}}
\expandafter\def\csname PY@tok@si\endcsname{\let\PY@bf=\textbf\def\PY@tc##1{\textcolor[rgb]{0.73,0.40,0.53}{##1}}}
\expandafter\def\csname PY@tok@se\endcsname{\let\PY@bf=\textbf\def\PY@tc##1{\textcolor[rgb]{0.73,0.40,0.13}{##1}}}
\expandafter\def\csname PY@tok@sr\endcsname{\def\PY@tc##1{\textcolor[rgb]{0.73,0.40,0.53}{##1}}}
\expandafter\def\csname PY@tok@ss\endcsname{\def\PY@tc##1{\textcolor[rgb]{0.10,0.09,0.49}{##1}}}
\expandafter\def\csname PY@tok@sx\endcsname{\def\PY@tc##1{\textcolor[rgb]{0.00,0.50,0.00}{##1}}}
\expandafter\def\csname PY@tok@m\endcsname{\def\PY@tc##1{\textcolor[rgb]{0.40,0.40,0.40}{##1}}}
\expandafter\def\csname PY@tok@gh\endcsname{\let\PY@bf=\textbf\def\PY@tc##1{\textcolor[rgb]{0.00,0.00,0.50}{##1}}}
\expandafter\def\csname PY@tok@gu\endcsname{\let\PY@bf=\textbf\def\PY@tc##1{\textcolor[rgb]{0.50,0.00,0.50}{##1}}}
\expandafter\def\csname PY@tok@gd\endcsname{\def\PY@tc##1{\textcolor[rgb]{0.63,0.00,0.00}{##1}}}
\expandafter\def\csname PY@tok@gi\endcsname{\def\PY@tc##1{\textcolor[rgb]{0.00,0.63,0.00}{##1}}}
\expandafter\def\csname PY@tok@gr\endcsname{\def\PY@tc##1{\textcolor[rgb]{1.00,0.00,0.00}{##1}}}
\expandafter\def\csname PY@tok@ge\endcsname{\let\PY@it=\textit}
\expandafter\def\csname PY@tok@gs\endcsname{\let\PY@bf=\textbf}
\expandafter\def\csname PY@tok@gp\endcsname{\let\PY@bf=\textbf\def\PY@tc##1{\textcolor[rgb]{0.00,0.00,0.50}{##1}}}
\expandafter\def\csname PY@tok@go\endcsname{\def\PY@tc##1{\textcolor[rgb]{0.53,0.53,0.53}{##1}}}
\expandafter\def\csname PY@tok@gt\endcsname{\def\PY@tc##1{\textcolor[rgb]{0.00,0.27,0.87}{##1}}}
\expandafter\def\csname PY@tok@err\endcsname{\def\PY@bc##1{\setlength{\fboxsep}{0pt}\fcolorbox[rgb]{1.00,0.00,0.00}{1,1,1}{\strut ##1}}}
\expandafter\def\csname PY@tok@kc\endcsname{\let\PY@bf=\textbf\def\PY@tc##1{\textcolor[rgb]{0.00,0.50,0.00}{##1}}}
\expandafter\def\csname PY@tok@kd\endcsname{\let\PY@bf=\textbf\def\PY@tc##1{\textcolor[rgb]{0.00,0.50,0.00}{##1}}}
\expandafter\def\csname PY@tok@kn\endcsname{\let\PY@bf=\textbf\def\PY@tc##1{\textcolor[rgb]{0.00,0.50,0.00}{##1}}}
\expandafter\def\csname PY@tok@kr\endcsname{\let\PY@bf=\textbf\def\PY@tc##1{\textcolor[rgb]{0.00,0.50,0.00}{##1}}}
\expandafter\def\csname PY@tok@bp\endcsname{\def\PY@tc##1{\textcolor[rgb]{0.00,0.50,0.00}{##1}}}
\expandafter\def\csname PY@tok@fm\endcsname{\def\PY@tc##1{\textcolor[rgb]{0.00,0.00,1.00}{##1}}}
\expandafter\def\csname PY@tok@vc\endcsname{\def\PY@tc##1{\textcolor[rgb]{0.10,0.09,0.49}{##1}}}
\expandafter\def\csname PY@tok@vg\endcsname{\def\PY@tc##1{\textcolor[rgb]{0.10,0.09,0.49}{##1}}}
\expandafter\def\csname PY@tok@vi\endcsname{\def\PY@tc##1{\textcolor[rgb]{0.10,0.09,0.49}{##1}}}
\expandafter\def\csname PY@tok@vm\endcsname{\def\PY@tc##1{\textcolor[rgb]{0.10,0.09,0.49}{##1}}}
\expandafter\def\csname PY@tok@sa\endcsname{\def\PY@tc##1{\textcolor[rgb]{0.73,0.13,0.13}{##1}}}
\expandafter\def\csname PY@tok@sb\endcsname{\def\PY@tc##1{\textcolor[rgb]{0.73,0.13,0.13}{##1}}}
\expandafter\def\csname PY@tok@sc\endcsname{\def\PY@tc##1{\textcolor[rgb]{0.73,0.13,0.13}{##1}}}
\expandafter\def\csname PY@tok@dl\endcsname{\def\PY@tc##1{\textcolor[rgb]{0.73,0.13,0.13}{##1}}}
\expandafter\def\csname PY@tok@s2\endcsname{\def\PY@tc##1{\textcolor[rgb]{0.73,0.13,0.13}{##1}}}
\expandafter\def\csname PY@tok@sh\endcsname{\def\PY@tc##1{\textcolor[rgb]{0.73,0.13,0.13}{##1}}}
\expandafter\def\csname PY@tok@s1\endcsname{\def\PY@tc##1{\textcolor[rgb]{0.73,0.13,0.13}{##1}}}
\expandafter\def\csname PY@tok@mb\endcsname{\def\PY@tc##1{\textcolor[rgb]{0.40,0.40,0.40}{##1}}}
\expandafter\def\csname PY@tok@mf\endcsname{\def\PY@tc##1{\textcolor[rgb]{0.40,0.40,0.40}{##1}}}
\expandafter\def\csname PY@tok@mh\endcsname{\def\PY@tc##1{\textcolor[rgb]{0.40,0.40,0.40}{##1}}}
\expandafter\def\csname PY@tok@mi\endcsname{\def\PY@tc##1{\textcolor[rgb]{0.40,0.40,0.40}{##1}}}
\expandafter\def\csname PY@tok@il\endcsname{\def\PY@tc##1{\textcolor[rgb]{0.40,0.40,0.40}{##1}}}
\expandafter\def\csname PY@tok@mo\endcsname{\def\PY@tc##1{\textcolor[rgb]{0.40,0.40,0.40}{##1}}}
\expandafter\def\csname PY@tok@ch\endcsname{\let\PY@it=\textit\def\PY@tc##1{\textcolor[rgb]{0.25,0.50,0.50}{##1}}}
\expandafter\def\csname PY@tok@cm\endcsname{\let\PY@it=\textit\def\PY@tc##1{\textcolor[rgb]{0.25,0.50,0.50}{##1}}}
\expandafter\def\csname PY@tok@cpf\endcsname{\let\PY@it=\textit\def\PY@tc##1{\textcolor[rgb]{0.25,0.50,0.50}{##1}}}
\expandafter\def\csname PY@tok@c1\endcsname{\let\PY@it=\textit\def\PY@tc##1{\textcolor[rgb]{0.25,0.50,0.50}{##1}}}
\expandafter\def\csname PY@tok@cs\endcsname{\let\PY@it=\textit\def\PY@tc##1{\textcolor[rgb]{0.25,0.50,0.50}{##1}}}

\def\PYZbs{\char`\\}
\def\PYZus{\char`\_}
\def\PYZob{\char`\{}
\def\PYZcb{\char`\}}
\def\PYZca{\char`\^}
\def\PYZam{\char`\&}
\def\PYZlt{\char`\<}
\def\PYZgt{\char`\>}
\def\PYZsh{\char`\#}
\def\PYZpc{\char`\%}
\def\PYZdl{\char`\$}
\def\PYZhy{\char`\-}
\def\PYZsq{\char`\'}
\def\PYZdq{\char`\"}
\def\PYZti{\char`\~}
% for compatibility with earlier versions
\def\PYZat{@}
\def\PYZlb{[}
\def\PYZrb{]}
\makeatother


    % Exact colors from NB
    \definecolor{incolor}{rgb}{0.0, 0.0, 0.5}
    \definecolor{outcolor}{rgb}{0.545, 0.0, 0.0}



    
    % Prevent overflowing lines due to hard-to-break entities
    \sloppy 
    % Setup hyperref package
    \hypersetup{
      breaklinks=true,  % so long urls are correctly broken across lines
      colorlinks=true,
      urlcolor=urlcolor,
      linkcolor=linkcolor,
      citecolor=citecolor,
      }
    % Slightly bigger margins than the latex defaults
    
    \geometry{verbose,tmargin=1in,bmargin=1in,lmargin=1in,rmargin=1in}
    
    

    \begin{document}
    
    
    \maketitle
    
    

    
    \section{Lecture 3}\label{lecture-3}

\subsection{August 28, 2018}\label{august-28-2018}

\subsection{Econ 501B}\label{econ-501b}

    \subsubsection{Theorem: Gale-Shapley
(1962):}\label{theorem-gale-shapley-1962}

\emph{In any one-to-one matching environment there exists a stable
matching \(\mu\).}

The proof is constructive such that it makes a stable matching \(\mu\)
such that it employs the T-proposal Deferred Acceptance Algorithm.

\textbf{Sketch of Algorithm}

\begin{enumerate}
\def\labelenumi{\arabic{enumi}.}
\tightlist
\item
  Each T agent proposes to her most preferred B agent if there is one
  that is acceptable to her. Each B agent tentatively holds on to the
  most preferred proposal provided that there is one acceptable to her.
\item
  Denote this step as \(k \ge 2\). Each T agent that was rejected at
  Step \(k-1\) makes the proposal to her next highest acceptable choice
  if there is one. And each B agent tentatively holds on to her most
  preferred option amongst new proposals plus \(k-1\) held proposals if
  there is one acceptable. All other proposals are rejected.
\end{enumerate}

Algorithm will terminate when: \(k = k+1\)

    Fix \(X \subset J\) where \(J\) is a set of agents. Write
\(\max_i X = \{j \in X: j \succsim_i j' \forall j' \in X\}\)

If \(X = \emptyset\) then \(\max_i X = \emptyset\)

    Formally, the Algorithm can be written as follows:

\textbf{Round 1:}

\(A^1(t) = A(t)\) T ask what agents are acceptable to me in round 1?

\(\hat{P}^1 : T \rightarrow B \cup \{\emptyset\} : \hat{P}^1 \in \max_t A^1(t)\)
I want to make proposal to best B agent. T is matched with best B agent
or keeps to himself (empty set).

If \(A^1(t) \ne \emptyset\) choose the best B agent. If
\(A^1(t) = \emptyset\), choose \(\emptyset\)

What are B agents going to do?

\(P^1(b) = (\hat{P}^1)^{-1}(\{b\}) \cap A(b) = \{t \in T: \hat{P}^1(t) = b : t \in A(b)\}\)

The above means t has made a proposal to me and that t is acceptable to
me.

    Now, \(\hat{\mu}^1: B \rightarrow T \cup \{\emptyset\}\) such that
\(\hat{\mu}^1 \in \max_b P^1(b)\)

If \(p^1(b) = \emptyset\) then \(\hat{\mu}^1(b) = \emptyset\). That is
if none are acceptable, I hold on to myself.

    \textbf{Round k+1}:

Inductively, I have defined:

\begin{itemize}
\tightlist
\item
  Sets \((A^k(t): t \in T)\)
\item
  maps \(\hat{P}^k: T \rightarrow B \cup \{\emptyset\}\)
\item
  sets \((P^k(b): b \in B)\) and \(P^k \subseteq T\)
\item
  maps \(\hat{\mu}^k: B \rightarrow T \cup \{\emptyset\}\)
\end{itemize}

    \(A^{k+1}(t) =\)

\textbf{(1)} \(A^k(t)\) if \(\hat{\mu}^k(\hat{P}^k) = t\) meaning t
proposed at k to \(b = \hat{P}^k(t)\) and b accepted t at \(k\).

But, he if rejected the offer, I throw him out of the running:

\textbf{(2)} \(A^k(t) \setminus \{\hat{P}^k(t)\}\)

if

\(\hat{\mu}^k(\hat{P}^k(t)) \ne t\)

If I was accepted on round \(k\), then I offer the same offer to the
same person on round \(k+1\).

\(\hat{P}^{k+1}(t) = \hat{P}^k\) if it is the case that
\(\hat{P}^k(t) \in A^{k+1}(t)\)

\(\hat{P}^{k+1}(t) \in \max_t A^{k+1}(t)\) if it is the case
\(\hat{P}^k(t) \notin A^{k+1}(t)\)

The set of proposals that b has on round \(k+1\):

\(P^{k+1}(b) = (\hat{P}^{k+1})^{-1} (\{b\}) \cap A(b)\)

Which is equivalent to:

\[\{t \in T: \hat{P}^{k+1}(t) = b: t \in A(b)\]

    I hold on to exactly the same offer if it is the case that b is still a
maximizer for me.

\[\hat{\mu}^{k+1}: B \rightarrow T \cup \{B\}\]

\(\hat{\mu}^{k+1}(b) = \hat{\mu}^{k}(b)\) if it is the case that
\(\hat{\mu}^{k+1}(b) \in \max_b P^{k+1}(b)\)

And

\(\hat{\mu}^{k+1}(b) = \max_b P^{k+1}(b)\) if it is the case that
\(\hat{\mu}^{k+1}(b) \notin \max_b P^{k+1}(b)\)

    \subsubsection{Lemma 1: the T-proposal in the algorithm
terminates:}\label{lemma-1-the-t-proposal-in-the-algorithm-terminates}

\[\exists K < \infty: \forall k \ge K: \hat{P}^k = \hat{P}^K \land \hat{\mu}^k = \hat{\mu}^K\]

    \textbf{Proof}

k is when the acceptable offer stops for every t agent.

For each \(t \in T\), \((A^k(t): k = 1,2,...)\) is decreasing.

\(..., \subseteq A^3(t) \subseteq A^2(t) \subseteq A^1(t)\)

\(\exists K: \forall t \in T, A^k(t) = A^K(t) \forall k \ge K\)

By definition, \(\hat{P}^k(t) = \hat{P}^k(t)\) for all \(t \in T\) and
\(k \ge K\)

This implies that \(\hat{P}^k(b) = \hat{P}^k(b)\) for all \(b \in B\)
and \(k \ge K\)

Which implies that \(\hat{\mu}^k(b) = \hat{\mu}^K(b)\) for all
\(b \in B\) and \(k \ge K\).

Show that the algorithm terminates. But does it give us a stable match?

    \subsubsection{Example}\label{example}

Let \(T = \{t_1, t_3, t_3\}\) and also \(B = \{b_1, b_2, b_3\}\)

\[t_1: b_2 \succ_{t_1} b_1 \succ_{t_1} b_3 \succ_{t_1} t_1\]

    \[t_2: b_1 \succ_{t_2} b_2 \succ_{t_2} b_3 \succ_{t_2} t_2\]

    \[t_3: b_1 \succ_{t_3} b_2 \succ_{t_3} b_3 \succ_{t_3} t_3\]

Notice \(t_2\) and \(t_3\) have the same preferences.

\(B\)'s preferences:

\[b_1: t_1 \succ_{b_1} t_3 \succ_{b_1} t_2 \succ_{b_1} b_1\]

    \[b_2: t_2 \succ_{b_2} t_1 \succ_{b_2} t_3 \succ_{b_2} b_2\]

    \[b_3: t_1 \succ_{b_3} t_3 \succ_{b_3} t_2 \succ_{b_3} b_3\]

Notice \(b_1\) and \(b_3\) have the same preference.

    \textbf{Step 1:}

\begin{itemize}
\tightlist
\item
  for all \(t \in T: A^1(t) = B\)
\item
  \(B^1(t_1) = b_2, \hat{P}^1(t_2) = b_1, \hat{P}^1(t_3) = b_1\)
\item
  \(P^1(b_1) = \{t_2, t_3\}\) and \(P^1(b_2) = \{t_1\}\) and
  \(P^1(b_3) = \emptyset\)
\item
  \(\hat{\mu}^1(b_1) = t_3\) and \(\hat{\mu}^1(b_2) = t_1\) and
  \(\hat{\mu}^1(b_3) = \emptyset\)
\end{itemize}

\textbf{Step 2:}

\begin{itemize}
\tightlist
\item
  \(A^2(t_1) = B\) and \(A^2(t_2) = \{b_2, b_3\}\) and \(A^2(t_3) = B\)
\item
  \(\hat{P}^2(t_1) = b_2\) and \(\hat{P}^2(t_2) = b_2\) and
  \(\hat{P}^2(t_3) = b_1\)
\item
  \(P^2(b_1) = \{t_3\}\) and \(P^2(b_2) = \{t_1, t_2\}\) and
  \(P^3(b_3) = \emptyset\)
\item
  \(\hat{\mu}^2(b_1) = t_3\) and \(\hat{\mu}^2(b_2) = t_2\) and
  \(\hat{\mu}^2(b_3) = \emptyset\)
\end{itemize}

\textbf{Step 3:}

\begin{itemize}
\tightlist
\item
  \(A^3(t_1) = \{b_1, b_3\}\) and \(A^3(t_2) = \{b_2, b_3\}\) and
  \(A^3(t_3) = B\)
\item
  \(\hat{P}^3(t_1) = b_1\) and \(\hat{P}^3(t_2) = b_2\) and
  \(\hat{P}^3(t_3) = b_1\)
\item
  \(P^3(b_1) = \{t_1, t_3\}\) and \(P^3(b_2) = \{t_2\}\) and
  \(P^3(b_3) = \emptyset\)
\item
  \(\hat{\mu}^3(b_1) = t_3\) and \(\hat{\mu}^3(b_2) = t_2\) and
  \(\hat{\mu}^3(b_3) = \emptyset\)
\end{itemize}

\textbf{Step 4:}

\begin{itemize}
\tightlist
\item
  \(A^4(t_1) = \{b_1, b_3\}\) and \(A^4(t_2) = \{b_2, b_3\}\) and
  \(A^4(t_3) = \{b_2, b_3\}\)
\item
  \(\hat{P}^4(t_1) = b_1\) and \(\hat{P}^4(t_2) = b_2\) and
  \(\hat{P}^4(t_3) = b_2\)
\item
  \(P^4(b_1) = \{t_1\}\) and \(P^4(b_2) = \{t_2, t_3\}\) and
  \(P^4(b_3) = \emptyset\)
\item
  \(\hat{\mu}^4(b_1) = t_1\) and \(\hat{\mu}^4(b_2) = t_2\) and
  \(\hat{\mu}^4(b_3) = \emptyset\)
\end{itemize}

\textbf{Step 5:}

\begin{itemize}
\tightlist
\item
  \(A^5(t_1) = \{b_1, b_3\}\) and \(A^5(t_2) = \{b_2, b_3\}\) and
  \(A^5(t_3) = \{b_3\}\)
\item
  \(\hat{P}^5(t_1) = b_1\) and \(\hat{P}^5(t_2) = b_2\) and
  \(\hat{P}^5(t_3) = b_3\)
\item
  \(P^5(b_1) = \{t_1\}\) and \(P^5(b_2) = \{t_2\}\) and
  \(P^5(b_3) = \{t_3\}\)
\item
  \(\hat{\mu}^5(b_1) = t_1\) and \(\hat{\mu}^5(b_2) = t_2\) and
  \(\hat{\mu}^5(b_3) = t_3\).
\end{itemize}

You can check this is exactly what was completed last class without the
Differed Acceptance Algorithm.

Since it terminates at \(k+1\) technically there is a \textbf{Step 6}
where everything in \textbf{5} is repeated.


    % Add a bibliography block to the postdoc
    
    
    
    \end{document}
